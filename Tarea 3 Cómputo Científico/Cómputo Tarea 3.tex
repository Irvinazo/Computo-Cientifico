\documentclass[letterpaper]{article} 
\usepackage[left = 0.5in, right = 0.5in, top = 0.9in, bottom = 0.9in]{geometry}
\usepackage{enumitem}
\usepackage{multicol}
\usepackage[spanish]{babel}
\usepackage[utf8]{inputenc}

\usepackage{amsmath,amssymb,amsthm}
\usepackage{tikz-cd}
\usepackage{mathrsfs}
\usepackage[bbgreekl]{mathbbol}
\usepackage{dsfont}
\newcommand{\op}{\operatorname}
\newcommand{\Op}{^{\op{op}}}
\newcommand{\scc}{\mathscr C}
\newcommand{\scd}{\mathscr D}
\newcommand{\sce}{\mathscr E}
\newcommand{\sci}{\mathscr I}
\newcommand{\scj}{\mathscr J}
\newcommand{\scx}{\mathscr X}
\newcommand{\var}{\mathrm{Var}}
\newcommand{\Id}{\operatorname{Id}}
\newcommand{\N}{\mathbb N}
\newcommand{\Z}{\mathbb Z}
\newcommand{\Q}{\mathbb{Q}}
\newcommand{\I}{\mathbb{I}}
\newcommand{\R}{\mathbb{R}}
\newcommand{\C}{\mathbb{C}}
\newcommand{\F}{\mathcal{F}}
\newcommand{\G}{\mathcal{G}}
\newcommand{\B}{\mathcal{B}}
\newcommand{\abs}[1]{\left\lvert #1 \right\rvert}
\newcommand{\inv}{^{-1}}
\renewcommand{\to}{\rightarrow}
\newcommand{\ent}{\Longrightarrow}
\newcommand{\E}{\mathbb{E}}
\renewcommand{\P}{\mathbb{P}}
\newcommand{\1}{\mathds{1}}
\renewcommand{\qedsymbol}{$\blacksquare$}

\theoremstyle{definition}
\newtheorem{dfn}{Definición}
\theoremstyle{definition}
\newtheorem{teo}{Teorema}
\theoremstyle{definition}
\newtheorem{cor}{Corolario}
\theoremstyle{definition}
\newtheorem{prop}{Proposición}
\theoremstyle{definition}
\newtheorem{obs}{Observación}


\title{\textbf{Cómputo Científico\\
Tarea 3\\   
Estabilidad}}
\author{Iván Irving Rosas Domínguez}
\date{\today}

\DeclareSymbolFontAlphabet{\mathbbm}{bbold}
\DeclareSymbolFontAlphabet{\mathbb}{AMSb}
\DeclareMathSymbol\bbDelta  \mathord{bbold}{"01}

\begin{document}
\maketitle

%\begin{abstract}
%\end{abstract}

\begin{enumerate}
    \item [\textbf{1.}] Sea $Q$ una matriz unitaria aleatoria de $20\times20$ (eg. con $A$ una matriz
    de tamaño $20\times 20$ aleatoria calculen su descomposición $QR$). Sean $\lambda_1>\lambda_2>...>\lambda_{20}=1>0$
    y 
    \[
    B=Q^*diag(\lambda_1,\lambda_2,...,\lambda_{20})Q \text{ y } B_\varepsilon=Q^*diag(\lambda_1+\varepsilon_1,\lambda_2+\varepsilon,...,\lambda_{20}+\varepsilon_20)Q,
    \]
    con $\varepsilon_i\sim N(0,\sigma)$, con  $\sigma=0.02$ y $\lambda_{20}=0.01$.
    \begin{enumerate}
        \item Comparar la descomposición de Cholesky de $B$ y de $B_\varepsilon$ usando 
        el algoritmo de la tarea 1. Considerar los casos cuando $B$ tiene un 
        $buen$ número de condición y un $mal$ número de condición
        \item Con el caso mal condicionado, comparar el resultado de su algoritmo con el del algoritmo
        de Cholesky de scipy.
        \item Medir el tiempo de ejecución de su algoritmo de Cholesky con el de scipy.
        \end{enumerate}

    \item [\textbf{2.}] Resolver el problema de mínimos cuadrados,
        \[
        y=X\beta+\varepsilon, \quad \varepsilon_i\sim N(0,\sigma),
        \]
        usando su implementación de la descomposición $QR$; $\beta$ es de tamaño $n\times 1$ y $X$ de tamaño
        $n\times d$. \\
        Sean $d=5$, $n=20$, $\beta=(5,4,3,2,1)^t$ y $\sigma=0.13$.
        \begin{enumerate}
            \item Hacer $X$ con entradas aleatorias $U(0,1)$ y simular $y$. Encontrar $\hat{\beta}$ y 
            compararlo con el obtenido $\hat{\beta}_p$ haciendo $X+\Delta X$, donde las entradas de
            $\Delta X$ son $N(0,\sigma)$,  $\sigma=0.01$. Comparar a su vez con $\hat{\beta}_c=\left(\left(X+\Delta X\right)^t\left(X+\Delta X\right)\right)^{-1}(X+\Delta X)^ty$,
            usando el algoritmo genérico para invertir matrices scipy.linalg.inv.
            \item Lo mismo que el anterior pero con $X$ mal condicionada (i.e. con casi colinealidad).
        \end{enumerate}
\end{enumerate}
\end{document}