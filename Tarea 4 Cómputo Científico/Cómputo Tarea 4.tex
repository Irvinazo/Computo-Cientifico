\documentclass[letterpaper]{article} 
\usepackage[left = 0.5in, right = 0.5in, top = 0.9in, bottom = 0.9in]{geometry}
\usepackage{enumitem}
\usepackage{multicol}
\usepackage[spanish]{babel}
\usepackage[utf8]{inputenc}

\usepackage{amsmath,amssymb,amsthm}
\usepackage{tikz-cd}
\usepackage{mathrsfs}
\usepackage[bbgreekl]{mathbbol}
\usepackage{dsfont}
\usepackage{graphicx}
\graphicspath{{img/}}

\newcommand{\op}{\operatorname}
\newcommand{\Op}{^{\op{op}}}
\newcommand{\scc}{\mathscr C}
\newcommand{\scd}{\mathscr D}
\newcommand{\sce}{\mathscr E}
\newcommand{\sci}{\mathscr I}
\newcommand{\scj}{\mathscr J}
\newcommand{\scx}{\mathscr X}
\newcommand{\var}{\mathrm{Var}}
\newcommand{\Id}{\operatorname{Id}}
\newcommand{\N}{\mathbb N}
\newcommand{\Z}{\mathbb Z}
\newcommand{\Q}{\mathbb{Q}}
\newcommand{\I}{\mathbb{I}}
\newcommand{\R}{\mathbb{R}}
\newcommand{\C}{\mathbb{C}}
\newcommand{\F}{\mathcal{F}}
\newcommand{\G}{\mathcal{G}}
\newcommand{\B}{\mathcal{B}}
\newcommand{\abs}[1]{\left\lvert #1 \right\rvert}
\newcommand{\inv}{^{-1}}
\renewcommand{\to}{\rightarrow}
\newcommand{\ent}{\Longrightarrow}
\newcommand{\E}{\mathbb{E}}
\renewcommand{\P}{\mathbb{P}}
\newcommand{\1}{\mathds{1}}
\renewcommand{\qedsymbol}{$\blacksquare$}

\theoremstyle{definition}
\newtheorem{dfn}{Definición}
\theoremstyle{definition}
\newtheorem{teo}{Teorema}
\theoremstyle{definition}
\newtheorem{cor}{Corolario}
\theoremstyle{definition}
\newtheorem{prop}{Proposición}
\theoremstyle{definition}
\newtheorem{obs}{Observación}


\title{\textbf{Cómputo Científico\\Tarea 4\\Cálculo de eigenvalores}}
\author{Iván Irving Rosas Domínguez}
\date{\today}

\DeclareSymbolFontAlphabet{\mathbbm}{bbold}
\DeclareSymbolFontAlphabet{\mathbb}{AMSb}
\DeclareMathSymbol\bbDelta  \mathord{bbold}{"01}

\begin{document}
\maketitle

%\begin{abstract}
%\end{abstract}

\begin{enumerate}
    \item[\textbf{1.}] Dado el siguiente:
    \begin{teo}[Gershgorin]
        Dada una matriz $A=(a_{ij})$ de $m\times m$, cada eigenvalor de A está en al menos
        uno de los discos en el plano complejo con centro en $a_{ii}$ y radio $\sigma_{j\neq i}|a_{ij}|$. 
        Además, si $n$ de estos discos forman un dominio conexo, disjunto de los otros $m-n$ discos, 
        entonces hay exactamente $n$ eigenvalores en ese dominio.
    \end{teo}
    Deduce estimaciones de los eigenvalores de 
\[
    \begin{pmatrix}
    8 & 1 & 0\\
    1 & 4 & \epsilon\\
    0 & \epsilon & 1\\    
    \end{pmatrix}
    \]
    con $|\epsilon|<1$.
    \item[\textbf{2.}] Implementa la iteración $QR$ con shift. Aplícala a la matriz $A$ del Ejercicio 1 con 
    $\epsilon=10^{-N}$ para $N=1,3,4,5$.
    \item[\textbf{3.}]Determina todos los eigenvalores y eigenvectores de una matriz de Householder
    \item[\textbf{4.}]Demuestra que no es posible construir la transformación de similaridad del Teorema
    de Schur con un número finito de transformaciones de similaridad de Householder.
    \item[\textbf{5.}]¿Qué pasa si aplicas la iteración $QR$ sin shift a una matriz ortogonal?
    o \textbf{hagan el que quieran} Sea A una matriz de Hessenberg superior y sea $QR=A$ la
    factorización $QR$ de $A$. Muestra que $RQ$ es una matriz superior de Hessenberg.
\end{enumerate}

\end{document}